\documentclass[11pt]{article}
\usepackage[utf8]{inputenc}
\usepackage{amssymb}
\usepackage{setspace}
\usepackage{graphicx}
\usepackage{hyperref}
\usepackage[table]{xcolor}

\addtolength{\oddsidemargin}{-.5in}
\addtolength{\evensidemargin}{-.5in}
\addtolength{\textwidth}{1.0in}

\addtolength{\topmargin}{-0.5in}
\addtolength{\textheight}{1.0in}

\def\urltilda{\kern -.15em\lower .7ex\hbox{\~{}}\kern .04em}

\begin{document}

\begin{center}

\textbf{\large{ECON 379:  Program Evaluation for International Development}} \\

\bigskip

\textbf{\large{Empirical Exercise \#3}} \\ 
Due 3/4 by 11:30 AM 

\end{center}

\bigskip

\bigskip

In this exercise, we will use the same data set as in Exercise 1, 
\texttt{E1-CohenEtAl-data.dta}, which is a subset of the data used in the paper 
``Price Subsidies, Diagnostic Tests, and Targeting of Malaria Treatment: 
Evidence from a Randomized Controlled Trial'' by Jessica Cohen, Pascaline 
Dupas, and Simone Schaner, published in the \emph{American Economic Review} in 2015. 

\bigskip 

In this exercise, we'll look at the way beliefs about how malaria is 
transmitted differ across households.  We'll review Stata's \texttt{generate} and \texttt{egen} commands, 
and we'll also practice applying the concepts covered in lecture and the 
readings, testing the equality of means across two groups using several 
different methods.

\bigskip

Create a do file containing the code below, and then use/modify it to answer the questions beginning at the bottom of the page:

\begin{verbatim}
// LOAD DATA

clear all 
set scheme s1mono 
set more off
set seed 314159
version 16.1

** change working directory as appropriate to where you want to save
cd "C:\Users\pj\Dropbox\econ379-2021\exercises\E3-false-counterfactuals"

** load the data from the course website
webuse set https://pjakiela.github.io/ECON379/exercises/E1-intro/
webuse E1-CohenEtAl-data.dta

** save the data locally (and then stop and save the do file as well)
save E3-CohenEtAl-raw-data, replace
\end{verbatim}

\bigskip

\noindent
Use the code above to help you answer the following questions:


\begin{enumerate}
	
\item Summarize the \texttt{b\_h\_edu} variable using the sum command.  What is the median 
	level of education among household heads in the sample?


\item Generate a dummy variable \texttt{med\_edu}  for having at least the median level of
	 education in the sample.  What is the mean of this dummy variable?


\item The variable \texttt{b\_knowledge\_correct} is a dummy variable equal to one for 
	if the respondent knows that malaria is transmitted by mosquitos.  
	What proportion of respondents know how malaria is transmitted?


\item What is the mean value of \texttt{b\_knowledge\_correct} among respondents who have 
	at least 6 years of education (ie at least the median level of education)?


\item The sum (or summarize) command tells you the standard deviation of the 
	mean of the \texttt{b\_knowledge\_correct} variable in the set of observations 
	with \texttt{med\_edu==1}, and it also tells you the number of observations with 
	\texttt{med\_edu==1}.  Use (only) these two pieces of information to calculate 
	the standard error (not the standard deviation) of the mean of the 
	\texttt{b\_knowledge\_correct} variable in the set of observations with \texttt{med\_edu==1}.
	What is this standard error?


\item Now use the ttest command to test whether the mean of \texttt{b\_knowledge\_correct} 
	is equal in the group with \texttt{med\_edu==1} and \texttt{med\_edu==0}.  Confirm that the 
	standard error that you calculated in Question 5 is correct.  Does the 
	mean value of \texttt{b\_knowledge\_correct} differ for respondents with above 
	versus below median eduction?  What is the t-statistic associated with 
	this hypothesis test?


\item What is the standard error associated with the difference in means (of 
	the \texttt{b\_knowledge\_correct} variable) between the \texttt{med\_edu==1} group and the 
	\texttt{med\_edu==0} group?  Calculate this by hand and convince yourself that you 
	could derive the same answer (up to rounding error) as the ttest command.


\item Regress \texttt{b\_knowledge\_correct} on the continuous \texttt{b\_h\_edu} variable.  What is 
	the OLS coefficient on \texttt{b\_h\_edu}?


\item Generate a variable equal to the mean of \texttt{b\_h\_edu}.  What is this mean?


\item Generate a variable equal to the difference between \texttt{b\_h\_edu} and its mean 
	(so, the value of this variable will be different for difference 
	observations). Call this variable \texttt{ed\_diff}. What is the mean of \texttt{ed\_diff}?


\item Generate a variable \texttt{y\_times\_ed\_diff} equal to \texttt{b\_knowledge\_correct} time 
	\texttt{ed\_diff}.  What is the mean of this variable?


\item Generate another variable \texttt{ed\_diff\_square} equal to \texttt{ed\_diff*ed\_diff}.  What 
	is the mean of this variable?


\item What is the ratio of the mean of \texttt{y\_times\_ed\_diff} to the mean of 
	\texttt{ed\_diff\_square}?  (Note that this is the same as the ratio of the *sum* 
	of all the values of \texttt{y\_times\_ed\_diff} to the *sum* of all the values of 
	\texttt{ed\_diff\_square}).


\item This number was the answer to one of the first 13 questions -- which one?

\end{enumerate}


\end{document}
