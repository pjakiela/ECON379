\documentclass[11pt]{article}
\usepackage[utf8]{inputenc}
\usepackage{amssymb}
\usepackage{setspace}
\usepackage{graphicx}
\usepackage{hyperref}
\usepackage[table]{xcolor}

\addtolength{\oddsidemargin}{-.5in}
\addtolength{\evensidemargin}{-.5in}
\addtolength{\textwidth}{1.0in}

\addtolength{\topmargin}{-0.5in}
\addtolength{\textheight}{1.0in}

\def\urltilda{\kern -.15em\lower .7ex\hbox{\~{}}\kern .04em}

\begin{document}

\begin{center}

\textbf{\large{ECON 379:  Program Evaluation for International Development}} \\

\bigskip

\textbf{\large{Empirical Exercise \#2}} \\ 
Due 3/2 by 11:30 AM 

\end{center}

\bigskip

\bigskip

In this exercise, we’ll use Stata’s \texttt{rnormal()} command to generate draws from a normally-distributed random variable. This approach -- simulating data according to a known data-generating process -- is an incredibly useful tool in empirical microeconomics (both for checking your econometric intuitions and your anlayis code).

\bigskip

We’ll use ``locals'' (also know as ``local macros'') to easily change the number of observations and other parameters of our data set. This will allow us to explore the way the properties of randomly-assigned treatment groups in larger and smaller samples.

\bigskip

This exercise introduces a range of practical coding tools: \texttt{rnormal()}, locals, and the \texttt{return list} and \texttt{display} commands. By varying the sample size, we’ll build a better understanding of the role that the Law of large Numbers plays in randomized evaluations.

\bigskip

Create a do file containing the code below, and then use/modify it to answer the questions on the next page:

\begin{verbatim}
// PRELIMINARIES

** start with a clean workspace
clear all
set more off // setting more off prevents your code from stopping halfway through
set seed 12345 // setting the seed makes pseudo-random draws replicable
set scheme s1mono // the scheme is only relevant when making graphs/figures
version 16.1 // make sure you use a specific version of Stata (for replicability)

** change working directory as appropriate to where you want to save
cd "C:\Users\pj\Dropbox\econ379-2021\exercises\E2-selection-bias"

** save your do file to a local directory no (do this by hand, not in code)


// GENERATE A DATA SET

** define a local that we'll use to indicate the number of observations
local myobs = 10

** use the localto create an empty data set with N=myobs rows
set obs `myobs'

** define some variables
gen y = rnormal()
gen z = 5*rnormal() + 10

** assign half the variables (observations 1 through N/2) to treatment
count
return list // this shows you all the local macros saved by your last command
local cutoff = (r(N)/1)/2
gen treatment = 1 in 1/`cutoff'
replace treatment = 0 if treatment==.
\end{verbatim}

\bigskip

\noindent
Use the code above to help you answer the following questions:


\begin{enumerate}
	
\item What is the mean of \texttt{y}?

\item What is the *variance* (not the standard deviation) of \texttt{y}?

\item What is the mean of \texttt{z}?

\item What is the variance of \texttt{z}?

\item Use the \texttt{ttest} command to test the hypothesis that the mean of \texttt{z} is 10.  What is the p-value?

\item Summarize the \texttt{treatment} variable:  what proportion of the data is assigned to treatment (between 0 and 1)?

\item Change the do file so that half the observations are assigned to treatment. Use \texttt{ttest} to test whether the mean of \texttt{z} differs between the treatment and control (ie untreated) groups.  What is the estimated difference in means?

\item What is the p-value associated with this difference?

\item Now change the number of observations to 1000 and rerun the do file.  What is the estimated difference in means?

\item What is the p-value associated with this difference?


\end{enumerate}


\end{document}
